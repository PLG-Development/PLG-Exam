\documentclass[]{article}
\usepackage[a4paper]{geometry}
\usepackage{graphicx}
\usepackage{amsmath}
\usepackage{float}
\usepackage{scrlayer-scrpage}
\usepackage{tabularx}
\usepackage{color}
\usepackage{enumitem}
\usepackage[hidelinks]{hyperref}
\renewcommand{\contentsname}{Inhalt}
\renewcommand{\figurename}{Grafik}
\title{Grundlagen zu PLG Exam}
\date{März 2025}
\author{Elias Fierke}
\begin{document}
	\begin{figure}
	\centering
		\includegraphics[width=0.2\textwidth]{plgicon.png}
	\label{fig:logo}
	\end{figure}
	\maketitle
	\tableofcontents
	\newpage
	\section{Einleitung}
	Herzlich Willkommen! In diesem Dokument lernen Sie die grundlegende Bedienung von PLG Exam, der schülerseitigen Schreibsoftware zum Lösen von Klausuren für Schüler*innen mit Nachteilsausgleich.\\\\
    Im Nachfolgenden wird an den meisten Stellen aufgrund von Lesbarkeit und fehlender Regelung in der deutschen Sprache nur "der Schüler" als Synonym für "Schülerinnen, Schüler, alles dazwischen und außenrum" verwendet. Gemeint ist also diejenige Person, die die Prüfung antritt und das Programm zum Bearbeiten der Aufgaben nutzt.
	
    \newpage
    \section{Start der Anwendung und erste Handlungen}
    Nachdem Sie den zugehörigen PC hochgefahren haben, starten Sie oder der Schüler die Anwendungen, entweder mittels Doppelklick auf die Datei "PLG-Exam", die sich im Hauptordner des Programms befindet, oder sie haben eine andere Möglichkeit, die mit dem zugehörigen Rechner-Beauftragten Ihrer Schule abgesprochen ist - zum Beispiel ein Autostart der Software oder ein auf dem Desktop liegendes Script, welches ebenfalls per Doppelklick ausgeführt werden kann.\\
    \subsection{Erste Handlungen}
    Nach dem Start der Software kann der Schüler 


\end{document}